\documentclass[11pt, a4paper]{article}
%\usepackage{geometry}
\usepackage[inner=1.5cm,outer=1.5cm,top=2.5cm,bottom=2.5cm]{geometry}
\pagestyle{empty}
\usepackage{graphicx}
\usepackage{fancyhdr, lastpage, bbding, pmboxdraw}
\usepackage[usenames,dvipsnames]{color}
\definecolor{darkblue}{rgb}{0,0,.6}
\definecolor{darkred}{rgb}{.7,0,0}
\definecolor{darkgreen}{rgb}{0,.6,0}
\definecolor{red}{rgb}{.98,0,0}
\usepackage[colorlinks,pagebackref,pdfusetitle,urlcolor=darkblue,citecolor=darkblue,linkcolor=darkred,bookmarksnumbered,plainpages=false]{hyperref}
\renewcommand{\thefootnote}{\fnsymbol{footnote}}

\pagestyle{fancyplain}
\fancyhf{}
\lhead{ \fancyplain{}{CSC2621} }
%\chead{ \fancyplain{}{} }
\rhead{ \fancyplain{}{\today} }
%\rfoot{\fancyplain{}{page \thepage\ of \pageref{LastPage}}}
\fancyfoot[RO, LE] {page \thepage\ of \pageref{LastPage} }
\thispagestyle{plain}

%%%%%%%%%%%% LISTING %%%
\usepackage{listings}
\usepackage{caption}
\DeclareCaptionFont{white}{\color{white}}
\DeclareCaptionFormat{listing}{\colorbox{gray}{\parbox{\textwidth}{#1#2#3}}}
\captionsetup[lstlisting]{format=listing,labelfont=white,textfont=white}
\usepackage{verbatim} % used to display code
\usepackage{fancyvrb}
\usepackage{acronym}
\usepackage{amsthm}
\VerbatimFootnotes % Required, otherwise verbatim does not work in footnotes!



\definecolor{OliveGreen}{cmyk}{0.64,0,0.95,0.40}
\definecolor{CadetBlue}{cmyk}{0.62,0.57,0.23,0}
\definecolor{lightlightgray}{gray}{0.93}



\lstset{
  %language=bash,                          % Code langugage
  basicstyle=\ttfamily,                   % Code font, Examples: \footnotesize, \ttfamily
  keywordstyle=\color{OliveGreen},        % Keywords font ('*' = uppercase)
  commentstyle=\color{gray},              % Comments font
  numbers=left,                           % Line nums position
  numberstyle=\tiny,                      % Line-numbers fonts
  stepnumber=1,                           % Step between two line-numbers
  numbersep=5pt,                          % How far are line-numbers from code
  backgroundcolor=\color{lightlightgray}, % Choose background color
  frame=none,                             % A frame around the code
  tabsize=2,                              % Default tab size
  captionpos=t,                           % Caption-position = bottom
  breaklines=true,                        % Automatic line breaking?
  breakatwhitespace=false,                % Automatic breaks only at whitespace?
  showspaces=false,                       % Dont make spaces visible
  showtabs=false,                         % Dont make tabls visible
  columns=flexible,                       % Column format
  morekeywords={__global__, __device__},  % CUDA specific keywords
}

%%%%%%%%%%%%%%%%%%%%%%%%%%%%%%%%%%%%
\begin{document}
\begin{center}
  {\Large \textsc{CSC2621 -- Imitation Learning for Robotics}}
\end{center}
\begin{center}
  Winter 2019
\end{center}
%\date{September 26, 2014}

\begin{center}
  \rule{6in}{0.4pt}
  \begin{minipage}[t]{.75\textwidth}
    \begin{tabular}{llcccll}
      \textbf{Instructor:} & Florian Shkurti & & &  & \textbf{Lectures:} Fri, 1-3pm, AB107\\
      \textbf{Email:} &  {florian@cs.toronto.edu} & & & & \textbf{Office Hours:} Tue, 4-6pm, PT283E\\
      \textbf{TA:} &  {TBA} & & & & \textbf{TA Office Hours:} TBA\\

    \end{tabular}
  \end{minipage}
  \rule{6in}{0.4pt}
\end{center}
\vspace{.5cm}
\setlength{\unitlength}{1in}
\renewcommand{\arraystretch}{2}

\noindent\textbf{Course Page:} \url{http://www.cs.toronto.edu/~florian/courses/imitation_learning}

\vskip.15in
\noindent\textbf{Overview:} %\footnotemark
This graduate-level seminar course will examine some of the most important papers in imitation learning for robot control, placing more emphasis on developments in the last 10 years. Its purpose is to familiarize students with the frontiers of this research area, to help them identify open problems, and to enable them to make a novel contribution. The majority of lectures, particularly after the first two weeks of introductory material, will consist of in-class student presentations. This course will broadly cover the following areas:

\begin{itemize}
\item Imitating the policies of demonstrators (people, expensive algorithms, optimal controllers)
\item Connections between imitation learning, optimal control, and reinforcement learning
\item Learning the cost functions that best explain a set of demonstrations
\item Shared autonomy between humans and robots for real-time control
\end{itemize}

\noindent The course involves a significant final project component, which will likely involve the use of robot simulators. Students who are interested in using real robot hardware and have shown sufficient progress in their final project, are encouraged to talk to the instructor about how to best arrange this.

\vskip.15in
\noindent\textbf{Prerequisites:} %\footnotemark
You need to be comfortable with introductory machine learning concepts (such as from CSC411/ECE521 or equivalent), linear algebra, basic multivariate calculus, intro to probability. You also need to have strong programming skills in Python. Note: if you don't meet all the prerequisites above please contact the instructor by email. Optional, but recommended: experience with neural networks, such as from CSC321 or equivalent, and introductory-level familiarity with reinforcement learning and control.

\vskip.15in
\noindent\textbf{Main References:} %\footnotemark
There is no required textbook for this course. In-class discussions will be based on research papers. The following are optional, but recommended textbooks:

\begin{itemize}
\item Aude Billard, Sylvain Calinon, Rudiger Dillmann, Stefan Schaal, {\textit{Robot programming by demonstration}}.
\item Sonia Chernova, Andrea Thomaz, {\textit{Robot learning from human teachers}}.
\item Takayuki Osa, Joni Pajarinen, Gerhard Neumann, Andrew Bagnell, Pieter Abbeel, Jan Peters, {\textit{An algorithmic perspective on imitation learning}}
\end{itemize}

\vskip.15in
\noindent\textbf{Course Communications:} %\footnotemark
\begin{itemize}
\item The official discussion board for the course is Quercus \url{https://q.utoronto.ca}. Announcements will be posted there, too.
\item Email the instructor or the TA with “CSC2621” in the subject line
\item You are welcome to provide anonymous feedback / suggestions for improvement any time during the semester: \url{https://www.surveymonkey.com/r/LJJV5LY}
\end{itemize}

\vspace*{.15in}
\noindent\textbf{Grading Policy:} 1x assignment (20\%),  1x paper presentation (20\%), and 1x course project (60\%). The grade of the course project consists of a proposal (10\%), midterm progress report (10\%), project presentation (10\%), and a final report with code at the end of the term (30\%).

% \footnotetext{Downloadable ebook versions are available on AeLP.}
\vspace*{.15in}

\noindent \textbf{Tentative Course Outline By Week:}
\begin{center}
  \begin{minipage}{5in}
    \begin{flushleft}
      %Chapter 1 \dotfill ~$\approx$ 3 days \\
      \begin{enumerate}
      \item Imitation vs. Robust Behavioral Cloning
      \item Intro to Optimal Control and Model-Based Reinforcement Learning
      \item Query-Efficient Policy Imitation via Novel State Detection
      \item Imitation Learning Combined with Reinforcement Learning, Control, and Planning \#1
      \item Imitation Learning Combined with Reinforcement Learning, Control, and Planning \#2
      \item Imitation as Program Induction and Modular Decomposition of Demonstrations
      \item Reading Week
      \item Inverse Reinforcement Learning \#1
      \item Inverse Reinforcement Learning \#2
      \item Shared Autonomy for Robot Control with Human in-the-Loop
      \item Adversarial Imitation Learning
      \item Project Presentations \#1
      \item Project Presentations \#2
      \end{enumerate}
    \end{flushleft}
  \end{minipage}
\end{center}



\vskip.15in
\noindent\textbf{Important Due Dates (tentative):}
\begin{center} \begin{minipage}{3.8in}
    \begin{flushleft}
      Assignment     \dotfill ~Feb 1, 2019, by 6pm EST\\
      Project Proposal      \dotfill ~Feb 6, 2019, by 6pm EST\\
      Reading Week      \dotfill ~Feb 18-22, 2019\\


      Midterm Project Report       \dotfill ~Mar 4, 2019, by 6pm EST  \\
      Project Presentations \dotfill ~Mar 29 and Apr 5, 2019 \\
      Project Report and Code \dotfill ~Apr 10, 2019, by 6pm EST\\

    \end{flushleft}
  \end{minipage}
\end{center}

\end{itemize}

\vskip.15in
\noindent\textbf{Class Attendance Policy:} Regular attendance and class participation with questions is essential. I expect you to be engaged and actively contribute to in-class discussions.

%%%%%% THE END
\end{document} 
